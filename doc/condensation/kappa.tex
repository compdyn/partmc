\documentclass[12pt]{article}
\usepackage{geometry} % see geometry.pdf on how to lay out the page. There's lots.
\geometry{a4paper} % or letter or a5paper or ... etc
% \geometry{landscape} % rotated page geometry

% See the ``Article customise'' template for come common customisations

\title{Single parameter approach accounting for the solubility and insolubility in an internally mixed aerosol particle}
\author{Joseph Ching}
%\date{} % delete this line to display the current date

%%% BEGIN DOCUMENT
\begin{document}

\maketitle
\tableofcontents

This short report is about the derivation of the relationship the single parameter proposed by number of previous studies (which account for the hygroscopicity) and the solute term in traditional solute term (Raoult) in Kohler's equation.


\section{Treatment of hygroscopicity, $\nu$ and $\kappa$}  
 
\begin{itemize}

\item An internally mixed aerosol particle could be a mulit-component system containing soluble inorganic salts, soluble organic substances, slightly soluble substances whose dissolution depend the amount of water in the aerosol particle, and also insoluble substances, e.g. soot.

\item When the solute effect is considered regarding the cloud droplet growth, the degree of dissociation of the inorganic salt is one of the parameters needed to be known. 

\item Traditionally, vant' Hoff factor, $i$ is used to quantify the degree of dissociation for inorganic salt. 

\item Some past literatures used, $\nu$ to quantify the number of ions produces per dissociation of one unit of salt. There is correspondence between $i$ and $\nu$ which will be also reviewed in this repot.

\item  In the case of NaCl, $\nu$ is 2, $\rm (NH_4)_2SO_4$, $\nu$ is 3.

\item Recent studies show that some of the organic substance existing in the atmosphere can also dissolve in water,

\item Hence, organics contribute to the Raoult term, which describes the solute effect in the and should be included in the Kohler's  framework. 

\item For organics, some studies suggest that $\kappa$ is used to quantify the hygroscopic behavior. 


\item In order to obtain a physical quantity to indicate overall hygroscopicity of an internally mixed aerosol particle, we can either convert $\nu$ of inorganic substances to $\kappa$ or convert $\kappa$ for organic substances to $\nu$. 


\item We choose the former approach. After obtaining a formulation of $\kappa$ for individual inorganic salt, we seek an overall $\kappa$ for a mixture consisting of pure inorganic substances. 

\item The ultimate goal is to obtain an overall $\kappa$ for a general internally mixed aerosol particle, which can contain soluble or insoluble; organic or inorganic substances.

\item Petters and Kreidenweis (2007) propose a single parameter, $\kappa$, representation of hygroscopic growth and condensation nucleus activity.

\item However they did not consider an aerosol particle containing insoluble substances in their theoretical framework.

\item The relationship between the solute in Kohler's equation and $\kappa$ is derived for the cases of 1) no insoluble substance contained 2) some insoluble substances contained.  

\item In Summary, we need to have a formulation for an overall $\kappa$ for a general mixture of soluble and insoluble substances and also organic and inorganic substance. 
 
\end{itemize}
 
\subsection{Petters and Kreidenweis (2007) $\kappa$ formulation (for aerosol particles contain no insoluble substances)}


\subsection{the relationship between $i$ and $\nu$}

\subsection{few concepts needed to be clarify}

\begin{itemize}

\item $i$ and $\nu$
\item water activity, $a_{w}$
\item water activity coefficient $\gamma_{w}$
\item partial molar volume of water $\overline{v_{w}}$
\item need to find out about whether $a_{w}$ change when the solution contain insoluble or slightly soluble substances.

\end{itemize}

\subsection{Petters and Kreidenweis (2007) $\kappa$ formulation Updated formulation on 11 March 2009}

Define $a_{w}$ as follows (from equation (2) in Kredenweis 2007)

\begin{equation}\label{eqn:1}
a_{w}^{-1}=1+\kappa \frac{V_s}{V_w}
\end{equation}

and equation (1) in Kredenweis 2007 states that

\begin{equation}
S=a_{w} \exp (\frac{4 \sigma_{s/a} M_{w}} {RT\rho_wD}).
\end{equation}

and note that also the another definition of $a_{w}$ from equation 15.18 in S\&P

\begin{equation}
S=\gamma_{w}(1+ \frac{n_{i}}{n_{w}})^{-1}\exp (\frac{4 \sigma_{s/a} M_{w}} {RT\rho_wD})
\end{equation}
where $n_{i}$ is the number of mole of ions after dissociation.

Comparing the above two definitions, we have

\begin{equation}
a_{w}=\gamma_{w}(1+\frac{n_{i}}{n_{w}})^{-1}=(1+\kappa \frac{V_{s}}{V_{w}})^{-1}.
\end{equation}

Then we have a relationship for $\kappa$ as follows, assuming $\gamma_{w}$=1, 

\begin{equation}
\kappa= \frac{V_{w} n_{i}}{V_{s} n_{w}}.
\end{equation}

Here is the relationship between $V_{s}$ and $n_{i}$, based on equation (15.33) on S$\&$P,


\begin{equation}
n_{i}=n_{s}= \frac {\nu \pi d_{s}^{3} \rho_{s} } {6 M_{s}},
\end{equation}
where $n_{s}$ is used in S\&P to stand for after dissociation and $\nu$ is number of ions produced per one  dissociation of molecules.

\subsection{aerosol particles containing insoluble substances}

For aerosol particle does not contain any insoluble substances,

\begin{equation}
\frac{1}{6} \pi D_{p}^{3}=n_{w} \overline{v_{w}} + n_{s} \overline{v_{s}}
\end{equation}

\begin{equation}
1+\frac{n_{i}}{n_{w}}= 1+ \frac{n_{i} \overline{v_{w}}} {(\pi / 6) D_{p}^{3} - n_{s} \overline{v_{s}} }
\end{equation}

For aerosol particle contain insoluble substances of diameter $d_{u}$

\begin{equation}
\frac{1}{6} \pi D_{p}^{3}=n_{w} \overline{v_{w}} + n_{s} \overline{v_{s}} + \frac{1}{6}\pi d_{u}^{3}
\end{equation}

\begin{equation}
1+\frac{n_{i}}{n_{w}}= 1+ \frac{n_{i} \overline{v_{w}}} {(\pi / 6) (D_{p}^{3}-d_{u}^{3} )- n_{s} \overline{v_{s}} }
\end{equation}



\subsection{relationship between $\kappa$ of individual component and overall $\kappa$}


Define $\kappa^{a}$ for individual component as follows, 

\begin{equation}
a_{w}^{-1}=1+\kappa^{a} \frac{V_{s}^{a}}{V_{w}^{a}} 
\end{equation}

\begin{equation}
a_{w}^{-1}=\gamma^{-1} (1+\frac{n_{i}^{a}}{n_{w}^{a}})
\end{equation}

assuming $\gamma_{w}$=1, we have 

\begin{equation}\label{eqn:13}
a_{w}^{-1}=(1+\frac{n_{i}^{a}}{n_{w}^{a}})
\end{equation}

By equating eqn(\ref{eqn:1}) to eqn(\ref{eqn:13}), we have

\begin{equation}
\kappa \frac{V_{s}}{V_{w}}= \kappa^{a} \frac{V_{s}^{a}}{V_{w}^{a}}
\end{equation}

\begin{equation}
\sum_{a} \kappa \frac{V_{s}}{V_{w}} V_{w}^{a}= \sum_{a} \kappa^{a} V_{s}^{a}
\end{equation}

\begin{equation}
\kappa V_{s}= \sum_{a} \kappa^{a} V_{s}^{a}
\end{equation}

\begin{equation}
\kappa= \sum_{a} \kappa^{a} \frac {V_{s}^{a}}{V_{s}} = \sum_{a} \kappa^{a} \epsilon_{a}
\end{equation}

\end{document}



