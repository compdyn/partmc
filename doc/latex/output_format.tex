Part\+MC output files are in the \href{http://www.unidata.ucar.edu/software/netcdf/}{\tt Net\+C\+DF Classic Format} (also known as Net\+C\+D\+F-\/3 format). The dimensions and variables in the files will depend on the type of run (particle, analytical solution, etc), and options in the spec file (e.\+g. {\ttfamily record\+\_\+removals} and {\ttfamily do\+\_\+optical}).

The state of the simulation is periodically output during the run, with frequency determined by the {\ttfamily t\+\_\+output} input parameter. Each output file has a filename of the form {\ttfamily P\+R\+E\+F\+I\+X\+\_\+\+R\+R\+R\+R\+\_\+\+S\+S\+S\+S\+S\+S\+S\+S.\+nc}, where {\ttfamily P\+R\+E\+F\+IX} is given by the {\ttfamily output\+\_\+prefix} input parameter, {\ttfamily R\+R\+RR} is the four-\/digit repeat number (starting from 1), and {\ttfamily S\+S\+S\+S\+S\+S\+SS} is the eight-\/digit output index (starting at 1 and incremented each time the state is output). For exact and sectional simulations all repeats would be identical so there is no support for repeating and the filename is of the format {\ttfamily P\+R\+E\+F\+I\+X\+\_\+\+S\+S\+S\+S\+S\+S\+S\+S.\+nc}.

If run in parallel and {\ttfamily output\+\_\+type} is {\ttfamily central} or {\ttfamily dist}, then the output files have names like {\ttfamily P\+R\+E\+F\+I\+X\+\_\+\+R\+R\+R\+R\+\_\+\+P\+P\+P\+P\+\_\+\+S\+S\+S\+S\+S\+S\+S\+S.\+nc}, where {\ttfamily P\+P\+PP} is a four-\/digit process number (starting from 1) and the other variables are as above. If {\ttfamily output\+\_\+type} is {\ttfamily single} then the output file naming scheme as the same as for serial runs.

The data in each output file comes in several different groups, as follows\+:

\mbox{\hyperlink{output_format_general}{General Information}}

\mbox{\hyperlink{output_format_env_state}{Environment State}}

\mbox{\hyperlink{output_format_gas_data}{Gas Material Data}}

\mbox{\hyperlink{output_format_gas_state}{Gas State}}

\mbox{\hyperlink{output_format_aero_data}{Aerosol Material Data}}

\mbox{\hyperlink{output_format_aero_state}{Aerosol Particle State}} (only for particle-\/resolved simulations)

\mbox{\hyperlink{output_format_aero_removed}{Aerosol Particle Removal Information}} (only for particle-\/resolved simulations, if {\ttfamily record\+\_\+removals} is {\ttfamily yes})

\mbox{\hyperlink{output_format_aero_weight_array}{Aerosol Weighting Function}} (only for particle-\/resolved simulations)

\mbox{\hyperlink{output_format_diam_bin_grid}{Diameter Bin Grid Data}} (only for exact and sectional simulations)

\mbox{\hyperlink{output_format_aero_binned}{Aerosol Binned Sectional State}} (only for exact and sectional simulations) \hypertarget{output_format_general}{}\section{Output File Format\+: General Information}\label{output_format_general}
Write the current state for a single process. Do not call this subroutine directly, but rather call \mbox{\hyperlink{namespacepmc__output_a3450534dae32c9b056de1292c80e0472}{output\+\_\+state()}}.


\begin{DoxyParams}[1]{Parameters}
\mbox{\tt in}  & {\em prefix} & Prefix of state file.\\
\hline
\mbox{\tt in}  & {\em aero\+\_\+data} & Aerosol data.\\
\hline
\mbox{\tt in}  & {\em aero\+\_\+state} & Aerosol state.\\
\hline
\mbox{\tt in}  & {\em gas\+\_\+data} & Gas data.\\
\hline
\mbox{\tt in}  & {\em gas\+\_\+state} & Gas state.\\
\hline
\mbox{\tt in}  & {\em env\+\_\+state} & Environment state.\\
\hline
\mbox{\tt in}  & {\em index} & Filename index.\\
\hline
\mbox{\tt in}  & {\em time} & Current time (s).\\
\hline
\mbox{\tt in}  & {\em del\+\_\+t} & Current timestep (s).\\
\hline
\mbox{\tt in}  & {\em i\+\_\+repeat} & Current repeat number.\\
\hline
\mbox{\tt in}  & {\em record\+\_\+removals} & Whether to output particle removal info.\\
\hline
\mbox{\tt in}  & {\em record\+\_\+optical} & Whether to output aerosol optical properties.\\
\hline
\mbox{\tt in}  & {\em uuid} & U\+U\+ID of the simulation.\\
\hline
\mbox{\tt in}  & {\em write\+\_\+rank} & Rank to write into file.\\
\hline
\mbox{\tt in}  & {\em write\+\_\+n\+\_\+proc} & Number of processes to write into file.\\
\hline
\end{DoxyParams}
The general information global Net\+C\+DF attributes are\+:
\begin{DoxyItemize}
\item {\bfseries title\+:} always set to the string \char`\"{}\+Part\+M\+C version V.\+V.\+V
     output file\char`\"{} where V.\+V.\+V is the Part\+MC version that created the file
\item {\bfseries source\+:} set to the string \char`\"{}\+Part\+M\+C version V.\+V.\+V\char`\"{}
\item {\bfseries U\+U\+ID\+:} a string of the form F47\+A\+C10\+B-\/58\+C\+C-\/4372-\/\+A567-\/0\+E02\+B2\+C3\+D479 which is the same for all files generated by a single call of Part\+MC.
\item {\bfseries Conventions\+:} set to the string \char`\"{}\+C\+F-\/1.\+4\char`\"{}, indicating compliance with the \href{http://cf-pcmdi.llnl.gov/documents/cf-conventions/1.4}{\tt CF convention format}
\item {\bfseries history\+:} set to the string \char`\"{}\+Y\+Y\+Y\+Y-\/\+M\+M-\/\+D\+D\+Thh\+:mm\+:ss\mbox{[}+-\/\mbox{]}\+Z\+Z\+:zz
     created by Part\+M\+C version V.\+V.\+V\char`\"{} where the first term is the file creation time in the \href{http://en.wikipedia.org/wiki/ISO_8601}{\tt I\+SO 8601 format}. For example, noon Pacific Standard Time (P\+ST) on February 1st, 2000 would be written 2000-\/02-\/01\+T12\+:00\+:00-\/08\+:00. The date and time variables are\+:
\begin{DoxyItemize}
\item Y\+Y\+YY\+: four-\/digit year
\item MM\+: two-\/digit month number
\item DD\+: two-\/digit day within month
\item T\+: literal \char`\"{}\+T\char`\"{} character
\item hh\+: two-\/digit hour in 24-\/hour format
\item mm\+: two-\/digit minute
\item ss\+: two-\/digit second
\item \mbox{[}+-\/\mbox{]}\+: a literal \char`\"{}+\char`\"{} or \char`\"{}-\/\char`\"{} character giving the time zone offset sign
\item ZZ\+: two-\/digit hours of the time zone offset from U\+TC
\item zz\+: two-\/digit minutes of the time zone offset from U\+TC
\end{DoxyItemize}
\end{DoxyItemize}

The general information Net\+C\+DF variables are\+:
\begin{DoxyItemize}
\item {\bfseries time} (unit s)\+: time elapsed since the simulation start time, as specified in the \mbox{\hyperlink{output_format_env_state}{Output File Format\+: Environment State}} section
\item {\bfseries timestep} (unit s)\+: the current timestep size
\item {\bfseries repeat\+:} the repeat number of this simulation (starting from 1)
\item {\bfseries timestep\+\_\+index\+:} an integer that is 1 on the first timestep, 2 on the second timestep, etc.
\item {\bfseries process} (M\+PI only)\+: the process number (starting from 1) that output this data file
\item {\bfseries total\+\_\+processes} (M\+PI only)\+: the total number of processes involved in writing data (may be less than the total number of processes that computed the data) 
\end{DoxyItemize}\hypertarget{output_format_env_state}{}\section{Output File Format\+: Environment State}\label{output_format_env_state}
Write full state.


\begin{DoxyParams}[1]{Parameters}
\mbox{\tt in}  & {\em env\+\_\+state} & Environment state to write.\\
\hline
\mbox{\tt in}  & {\em ncid} & Net\+C\+DF file ID, in data mode.\\
\hline
\end{DoxyParams}
The environment state Net\+C\+DF variables are\+:
\begin{DoxyItemize}
\item {\bfseries temperature} (unit K)\+: current air temperature
\item {\bfseries relative\+\_\+humidity} (dimensionless)\+: current air relative humidity (value of 1 means completely saturated)
\item {\bfseries pressure} (unit Pa)\+: current air pressure
\item {\bfseries longitude} (unit degrees\+\_\+east)\+: longitude of simulation location
\item {\bfseries latitude} (unit degrees\+\_\+north)\+: latitude of simulation location
\item {\bfseries altitude} (unit m)\+: altitude of simulation location
\item {\bfseries start\+\_\+time\+\_\+of\+\_\+day} (unit s)\+: time-\/of-\/day of the simulation start measured in seconds after midnight U\+TC
\item {\bfseries start\+\_\+day\+\_\+of\+\_\+year\+:} day-\/in-\/year number of the simulation start (starting from 1 on the first day of the year)
\item {\bfseries elapsed\+\_\+time} (unit s)\+: elapsed time since the simulation start
\item {\bfseries solar\+\_\+zenith\+\_\+angle} (unit radians)\+: current angle from the zenith to the sun
\item {\bfseries height} (unit m)\+: current boundary layer mixing height
\end{DoxyItemize}

See also\+:
\begin{DoxyItemize}
\item \mbox{\hyperlink{input_format_env_state}{Input File Format\+: Environment State}} and \mbox{\hyperlink{input_format_scenario}{Input File Format\+: Scenario}} --- the corresponding input formats 
\end{DoxyItemize}\hypertarget{output_format_gas_data}{}\section{Output File Format\+: Gas Material Data}\label{output_format_gas_data}
Write full state.


\begin{DoxyParams}[1]{Parameters}
\mbox{\tt in}  & {\em gas\+\_\+data} & Gas\+\_\+data to write.\\
\hline
\mbox{\tt in}  & {\em ncid} & Net\+C\+DF file ID, in data mode.\\
\hline
\end{DoxyParams}
The gas material data Net\+C\+DF dimensions are\+:
\begin{DoxyItemize}
\item {\bfseries gas\+\_\+species\+:} number of gas species
\end{DoxyItemize}

The gas material data Net\+C\+DF variables are\+:
\begin{DoxyItemize}
\item {\bfseries gas\+\_\+species} (dim {\ttfamily gas\+\_\+species})\+: dummy dimension variable (no useful value) -\/ read species names as comma-\/separated values from the \textquotesingle{}names\textquotesingle{} attribute
\item {\bfseries gas\+\_\+mosaic\+\_\+index} (dim {\ttfamily gas\+\_\+species})\+: M\+O\+S\+A\+IC indices of gas species
\end{DoxyItemize}

See also\+:
\begin{DoxyItemize}
\item \mbox{\hyperlink{input_format_gas_data}{Input File Format\+: Gas Material Data}} --- the corresponding input format 
\end{DoxyItemize}\hypertarget{output_format_gas_state}{}\section{Output File Format\+: Gas State}\label{output_format_gas_state}
Write full state.


\begin{DoxyParams}[1]{Parameters}
\mbox{\tt in}  & {\em gas\+\_\+state} & Gas state to write.\\
\hline
\mbox{\tt in}  & {\em ncid} & Net\+C\+DF file ID, in data mode.\\
\hline
\mbox{\tt in}  & {\em gas\+\_\+data} & Gas data.\\
\hline
\end{DoxyParams}
The gas state uses the {\ttfamily gas\+\_\+species} Net\+C\+DF dimension as specified in the \mbox{\hyperlink{output_format_gas_data}{Output File Format\+: Gas Material Data}} section.

The gas state Net\+C\+DF variables are\+:
\begin{DoxyItemize}
\item {\bfseries gas\+\_\+mixing\+\_\+ratio} (unit ppb, dim {\ttfamily gas\+\_\+species})\+: current mixing ratios of each gas species
\end{DoxyItemize}

See also\+:
\begin{DoxyItemize}
\item \mbox{\hyperlink{output_format_gas_data}{Output File Format\+: Gas Material Data}} --- the gas species list and material data output format
\item \mbox{\hyperlink{input_format_gas_state}{Input File Format\+: Gas State}} --- the corresponding input format 
\end{DoxyItemize}\hypertarget{output_format_aero_data}{}\section{Output File Format\+: Aerosol Material Data}\label{output_format_aero_data}
Write full state.


\begin{DoxyParams}[1]{Parameters}
\mbox{\tt in}  & {\em aero\+\_\+data} & Aero\+\_\+data to write.\\
\hline
\mbox{\tt in}  & {\em ncid} & Net\+C\+DF file ID, in data mode.\\
\hline
\end{DoxyParams}
The aerosol material data Net\+C\+DF dimensions are\+:
\begin{DoxyItemize}
\item {\bfseries aero\+\_\+species\+:} number of aerosol species
\item {\bfseries aero\+\_\+source\+:} number of aerosol sources
\end{DoxyItemize}

The aerosol material data Net\+C\+DF variables are\+:
\begin{DoxyItemize}
\item {\bfseries aero\+\_\+species} (dim {\ttfamily aero\+\_\+species})\+: dummy dimension variable (no useful value) -\/ read species names as comma-\/separated values from the \textquotesingle{}names\textquotesingle{} attribute
\item {\bfseries aero\+\_\+source} (dim {\ttfamily aero\+\_\+source})\+: dummy dimension variable (no useful value) -\/ read source names as comma-\/separated values from the \textquotesingle{}names\textquotesingle{} attribute
\item {\bfseries aero\+\_\+mosaic\+\_\+index} (dim {\ttfamily aero\+\_\+species})\+: indices of species in M\+O\+S\+A\+IC
\item {\bfseries aero\+\_\+density} (unit kg/m$^\wedge$3, dim {\ttfamily aero\+\_\+species})\+: densities of aerosol species
\item {\bfseries aero\+\_\+num\+\_\+ions} (dim {\ttfamily aero\+\_\+species})\+: number of ions produced when one molecule of each species fully dissociates in water
\item {\bfseries aero\+\_\+molec\+\_\+weight} (unit kg/mol, dim {\ttfamily aero\+\_\+species})\+: molecular weights of aerosol species
\item {\bfseries aero\+\_\+kappa} (unit kg/mol, dim {\ttfamily aero\+\_\+species})\+: hygroscopicity parameters of aerosol species
\item {\bfseries fractal} parameters, see \mbox{\hyperlink{output_format_fractal}{Output File Format\+: Fractal Data}}
\end{DoxyItemize}

See also\+:
\begin{DoxyItemize}
\item \mbox{\hyperlink{input_format_aero_data}{Input File Format\+: Aerosol Material Data}} --- the corresponding input format 
\end{DoxyItemize}\hypertarget{output_format_aero_state}{}\section{Output File Format\+: Aerosol Particle State}\label{output_format_aero_state}
Write full state.


\begin{DoxyParams}[1]{Parameters}
\mbox{\tt in}  & {\em aero\+\_\+state} & aero\+\_\+state to write.\\
\hline
\mbox{\tt in}  & {\em ncid} & Net\+C\+DF file ID, in data mode.\\
\hline
\mbox{\tt in}  & {\em aero\+\_\+data} & aero\+\_\+data structure.\\
\hline
\mbox{\tt in}  & {\em record\+\_\+removals} & Whether to output particle removal info.\\
\hline
\mbox{\tt in}  & {\em record\+\_\+optical} & Whether to output aerosol optical properties.\\
\hline
\end{DoxyParams}
The aerosol state consists of a set of individual aerosol particles, each with its own individual properties. The properties of all particles are stored in arrays, one per property. For example, {\ttfamily aero\+\_\+absorb\+\_\+cross\+\_\+sect(i)} gives the absorption cross section of particle number {\ttfamily i}, while {\ttfamily aero\+\_\+particle\+\_\+mass(i,s)} gives the mass of species {\ttfamily s} in particle {\ttfamily i}. The aerosol species are described in \mbox{\hyperlink{output_format_aero_data}{Output File Format\+: Aerosol Material Data}}.

Each aerosol particle {\ttfamily i} represents a number concentration given by {\ttfamily aero\+\_\+num\+\_\+conc(i)}. Multiplying a per-\/particle quantity by the respective number concentration gives the concentration of that quantity contributed by the particle. For example, summing {\ttfamily aero\+\_\+particle\+\_\+mass(i,s) $\ast$ aero\+\_\+num\+\_\+conc(i)} over all {\ttfamily i} gives the total mass concentration of species {\ttfamily s} in kg/m$^\wedge$3. Similarly, summing {\ttfamily aero\+\_\+absorb\+\_\+cross\+\_\+sect(i) $\ast$ aero\+\_\+num\+\_\+conc(i)} over all {\ttfamily i} will give the concentration of scattering cross section in m$^\wedge$2/m$^\wedge$3.

F\+I\+X\+ME\+: the aero\+\_\+weight is also output

The aerosol state uses the {\ttfamily aero\+\_\+species} Net\+C\+DF dimension as specified in the \mbox{\hyperlink{output_format_aero_data}{Output File Format\+: Aerosol Material Data}} section, as well as the Net\+C\+DF dimension\+:
\begin{DoxyItemize}
\item {\bfseries aero\+\_\+particle\+:} number of aerosol particles
\end{DoxyItemize}

The aerosol state Net\+C\+DF variables are\+:
\begin{DoxyItemize}
\item {\bfseries aero\+\_\+particle} (dim {\ttfamily aero\+\_\+particle})\+: dummy dimension variable (no useful value)
\item {\bfseries aero\+\_\+particle\+\_\+mass} (unit kg, dim {\ttfamily aero\+\_\+particle x aero\+\_\+species})\+: constituent masses of each aerosol particle -\/ {\ttfamily aero\+\_\+particle\+\_\+mass(i,s)} gives the mass of species {\ttfamily s} in particle {\ttfamily i} 
\item {\bfseries aero\+\_\+n\+\_\+orig\+\_\+part} (dim {\ttfamily aero\+\_\+particle x aero\+\_\+source})\+: number of original particles from each source that formed each aerosol particle -\/ {\ttfamily aero\+\_\+n\+\_\+orig\+\_\+part(i,s)} is the number of particles from source {\ttfamily s} that contributed to particle {\ttfamily i} -\/ when particle {\ttfamily i} first enters the simulation (by emissions, dilution, etc.) it has {\ttfamily aero\+\_\+n\+\_\+orig\+\_\+part(i,s) = 1} for the source number {\ttfamily s} it came from (otherwise zero) and when two particles coagulate, their values of {\ttfamily aero\+\_\+n\+\_\+orig\+\_\+part} are added (the number of coagulation events that formed each particle is thus {\ttfamily sum(aero\+\_\+n\+\_\+orig\+\_\+part(i,\+:)) -\/ 1})
\item {\bfseries aero\+\_\+particle\+\_\+weight\+\_\+group} (dim {\ttfamily aero\+\_\+particle})\+: weight group number (1 to {\ttfamily aero\+\_\+weight\+\_\+group}) of each aerosol particle
\item {\bfseries aero\+\_\+particle\+\_\+weight\+\_\+class} (dim {\ttfamily aero\+\_\+particle})\+: weight class number (1 to {\ttfamily aero\+\_\+weight\+\_\+class}) of each aerosol particle
\item {\bfseries aero\+\_\+absorb\+\_\+cross\+\_\+sect} (unit m$^\wedge$2, dim {\ttfamily aero\+\_\+particle})\+: optical absorption cross sections of each aerosol particle
\item {\bfseries aero\+\_\+scatter\+\_\+cross\+\_\+sect} (unit m$^\wedge$2, dim {\ttfamily aero\+\_\+particle})\+: optical scattering cross sections of each aerosol particle
\item {\bfseries aero\+\_\+asymmetry} (dimensionless, dim {\ttfamily aero\+\_\+particle})\+: optical asymmetry parameters of each aerosol particle
\item {\bfseries aero\+\_\+refract\+\_\+shell\+\_\+real} (dimensionless, dim {\ttfamily aero\+\_\+particle})\+: real part of the refractive indices of the shell of each aerosol particle
\item {\bfseries aero\+\_\+refract\+\_\+shell\+\_\+imag} (dimensionless, dim {\ttfamily aero\+\_\+particle})\+: imaginary part of the refractive indices of the shell of each aerosol particle
\item {\bfseries aero\+\_\+refract\+\_\+core\+\_\+real} (dimensionless, dim {\ttfamily aero\+\_\+particle})\+: real part of the refractive indices of the core of each aerosol particle
\item {\bfseries aero\+\_\+refract\+\_\+core\+\_\+imag} (dimensionless, dim {\ttfamily aero\+\_\+particle})\+: imaginary part of the refractive indices of the core of each aerosol particle
\item {\bfseries aero\+\_\+core\+\_\+vol} (unit m$^\wedge$3, dim {\ttfamily aero\+\_\+particle})\+: volume of the optical cores of each aerosol particle
\item {\bfseries aero\+\_\+water\+\_\+hyst\+\_\+leg} (dim {\ttfamily aero\+\_\+particle})\+: integers specifying which leg of the water hysteresis curve each particle is on, using the M\+O\+S\+A\+IC numbering convention
\item {\bfseries aero\+\_\+num\+\_\+conc} (unit m$^\wedge$\{-\/3\}, dim {\ttfamily aero\+\_\+particle})\+: number concentration associated with each particle
\item {\bfseries aero\+\_\+id} (dim {\ttfamily aero\+\_\+particle})\+: unique ID number of each aerosol particle
\item {\bfseries aero\+\_\+least\+\_\+create\+\_\+time} (unit s, dim {\ttfamily aero\+\_\+particle})\+: least (earliest) creation time of any original constituent particles that coagulated to form each particle, measured from the start of the simulation -\/ a particle is said to be created when it first enters the simulation (by emissions, dilution, etc.)
\item {\bfseries aero\+\_\+greatest\+\_\+create\+\_\+time} (unit s, dim {\ttfamily aero\+\_\+particle})\+: greatest (latest) creation time of any original constituent particles that coagulated to form each particle, measured from the start of the simulation -\/ a particle is said to be created when it first enters the simulation (by emissions, dilution, etc.) 
\end{DoxyItemize}\hypertarget{output_format_aero_removed}{}\section{Output File Format\+: Aerosol Particle Removal Information}\label{output_format_aero_removed}
When an aerosol particle is introduced into the simulation it is assigned a unique ID number. This ID number will persist over time, allowing tracking of a paticular particle\textquotesingle{}s evolution. If the {\ttfamily record\+\_\+removals} variable in the input spec file is {\ttfamily yes}, then the every time a particle is removed from the simulation its removal will be recorded in the removal information.

The removal information written at timestep {\ttfamily n} contains information about every particle ID that is present at time {\ttfamily (n -\/ 1)} but not present at time {\ttfamily n}.

The removal information is always written in the output files, even if no particles were removed in the previous timestep. Unfortunately, Net\+C\+DF files cannot contain arrays of length 0. In the case of no particles being removed, the {\ttfamily aero\+\_\+removed} dimension will be set to 1 and {\ttfamily aero\+\_\+removed\+\_\+action(1)} will be 0 ({\ttfamily A\+E\+R\+O\+\_\+\+I\+N\+F\+O\+\_\+\+N\+O\+NE}).

When two particles coagulate, the ID number of the combined particle will be the ID particle of the largest constituent, if possible (weighting functions can make this impossible to achieve). A given particle ID may thus be lost due to coagulation (if the resulting combined particle has a different ID), or the ID may be preserved (as the ID of the combined particle). Only if the ID is lost will the particle be recorded in the removal information, and in this case {\ttfamily aero\+\_\+removed\+\_\+action(i)} will be 2 ({\ttfamily A\+E\+R\+O\+\_\+\+I\+N\+F\+O\+\_\+\+C\+O\+AG}) and {\ttfamily aero\+\_\+removed\+\_\+other\+\_\+id(i)} will be the ID number of the combined particle.

The aerosol removal information Net\+C\+DF dimensions are\+:
\begin{DoxyItemize}
\item {\bfseries aero\+\_\+removed\+:} number of aerosol particles removed from the simulation during the previous timestep (or 1, as described above)
\end{DoxyItemize}

The aerosol removal information Net\+C\+DF variables are\+:
\begin{DoxyItemize}
\item {\bfseries aero\+\_\+removed} (dim {\ttfamily aero\+\_\+removed})\+: dummy dimension variable (no useful value)
\item {\bfseries aero\+\_\+removed\+\_\+id} (dim {\ttfamily aero\+\_\+removed})\+: the ID number of each removed particle
\item {\bfseries aero\+\_\+removed\+\_\+action} (dim {\ttfamily aero\+\_\+removed})\+: the reasons for removal for each particle, with values\+:
\begin{DoxyItemize}
\item 0 ({\ttfamily A\+E\+R\+O\+\_\+\+I\+N\+F\+O\+\_\+\+N\+O\+NE})\+: no information (invalid entry)
\item 1 ({\ttfamily A\+E\+R\+O\+\_\+\+I\+N\+F\+O\+\_\+\+D\+I\+L\+U\+T\+I\+ON})\+: particle was removed due to dilution with outside air
\item 2 ({\ttfamily A\+E\+R\+O\+\_\+\+I\+N\+F\+O\+\_\+\+C\+O\+AG})\+: particle was removed due to coagulation
\item 3 ({\ttfamily A\+E\+R\+O\+\_\+\+I\+N\+F\+O\+\_\+\+H\+A\+L\+V\+ED})\+: particle was removed due to halving of the aerosol population
\item 4 ({\ttfamily A\+E\+R\+O\+\_\+\+I\+N\+F\+O\+\_\+\+W\+E\+I\+G\+HT})\+: particle was removed due to adjustments in the particle\textquotesingle{}s weighting function
\end{DoxyItemize}
\item {\bfseries aero\+\_\+removed\+\_\+other\+\_\+id} (dim {\ttfamily aero\+\_\+removed})\+: the ID number of the combined particle formed by coagulation, if the removal reason was coagulation (2, {\ttfamily A\+E\+R\+O\+\_\+\+I\+N\+F\+O\+\_\+\+C\+O\+AG}). May be 0, if the new coagulated particle was not created due to weighting. 
\end{DoxyItemize}\hypertarget{output_format_aero_weight_array}{}\section{Output File Format\+: Aerosol Weighting Functions}\label{output_format_aero_weight_array}
Write full aero\+\_\+weight\+\_\+array.


\begin{DoxyParams}[1]{Parameters}
\mbox{\tt in}  & {\em aero\+\_\+weight\+\_\+array} & Aero weight array to write.\\
\hline
\mbox{\tt in}  & {\em ncid} & Net\+C\+DF file ID, in data mode.\\
\hline
\end{DoxyParams}
The aerosol weighting function Net\+C\+DF dimensions are\+:
\begin{DoxyItemize}
\item {\bfseries aero\+\_\+weight\+\_\+group\+:} number of aerosol weighting groups
\item {\bfseries aero\+\_\+weight\+\_\+class\+:} number of aerosol weighting classes
\end{DoxyItemize}

The aerosol weighting function Net\+C\+DF variables are\+:
\begin{DoxyItemize}
\item {\bfseries weight\+\_\+type} (no unit, dim {\ttfamily aero\+\_\+weight})\+: the type of each weighting function, with 0 = invalid weight, 1 = no weight ( $w(D) = 1$), 2 = power weight ( $w(D) = (D/D_0)^\alpha$), 3 = M\+FA weight ( $w(D) = (D/D_0)^{-3}$)
\item {\bfseries weight\+\_\+magnitude} (unit m$^\wedge$\{-\/3\}, dim {\ttfamily aero\+\_\+weight})\+: the number concentration magnitude associated with each weighting function
\item {\bfseries weight\+\_\+exponent} (no unit, dim {\ttfamily aero\+\_\+weight})\+: for each weighting function, specifies the exponent $\alpha$ for the power {\ttfamily weight\+\_\+type}, the value -\/3 for the M\+FA {\ttfamily weight\+\_\+type}, and zero for any other {\ttfamily weight\+\_\+type} 
\end{DoxyItemize}\hypertarget{output_format_diam_bin_grid}{}\section{Output File Format\+: Diameter Bin Grid Data}\label{output_format_diam_bin_grid}
The aerosol diameter bin grid data Net\+C\+DF dimensions are\+:


\begin{DoxyItemize}
\item {\bfseries aero\+\_\+diam\+:} number of bins (grid cells) on the diameter axis
\item {\bfseries aero\+\_\+diam\+\_\+edges\+:} number of bin edges (grid cell edges) on the diameter axis --- always equal to {\ttfamily aero\+\_\+diam + 1}
\end{DoxyItemize}

The aerosol diameter bin grid data Net\+C\+DF variables are\+:
\begin{DoxyItemize}
\item {\bfseries aero\+\_\+diam} (unit m, dim {\ttfamily aero\+\_\+diam})\+: aerosol diameter axis bin centers --- centered on a logarithmic scale from the edges, so that {\ttfamily aero\+\_\+diam(i) / aero\+\_\+diam\+\_\+edges(i) = sqrt(aero\+\_\+diam\+\_\+edges(i+1) / aero\+\_\+diam\+\_\+edges(i))}
\item {\bfseries aero\+\_\+diam\+\_\+edges} (unit m, dim {\ttfamily aero\+\_\+diam\+\_\+edges})\+: aersol diameter axis bin edges (there is one more edge than center)
\item {\bfseries aero\+\_\+diam\+\_\+widths} (dimensionless, dim {\ttfamily aero\+\_\+diam})\+: the base-\/e logarithmic bin widths --- {\ttfamily aero\+\_\+diam\+\_\+widths(i) = ln(aero\+\_\+diam\+\_\+edges(i+1) / aero\+\_\+diam\+\_\+edges(i))}, so all bins have the same width
\end{DoxyItemize}

See also\+:
\begin{DoxyItemize}
\item \mbox{\hyperlink{input_format_diam_bin_grid}{Input File Format\+: Diameter Axis Bin Grid}} --- the corresponding input format 
\end{DoxyItemize}\hypertarget{output_format_aero_binned}{}\section{Output File Format\+: Aerosol Binned Sectional State}\label{output_format_aero_binned}
Write full state.


\begin{DoxyParams}[1]{Parameters}
\mbox{\tt in}  & {\em aero\+\_\+binned} & Aero\+\_\+binned to write.\\
\hline
\mbox{\tt in}  & {\em ncid} & Net\+C\+DF file ID, in data mode.\\
\hline
\mbox{\tt in}  & {\em bin\+\_\+grid} & bin\+\_\+grid structure.\\
\hline
\mbox{\tt in}  & {\em aero\+\_\+data} & aero\+\_\+data structure.\\
\hline
\end{DoxyParams}
The aerosol size distributions (number and mass) are stored on a logarmithmic grid (see the \mbox{\hyperlink{output_format_diam_bin_grid}{Output File Format\+: Diameter Bin Grid Data}} section). To compute the total number or mass concentration, compute the sum over {\ttfamily i} of {\ttfamily aero\+\_\+number\+\_\+concentration(i) $\ast$ aero\+\_\+diam\+\_\+widths(i)}, for example.

The aerosol binned sectional state uses the {\ttfamily aero\+\_\+species} Net\+C\+DF dimension as specified in the \mbox{\hyperlink{output_format_aero_data}{Output File Format\+: Aerosol Material Data}} section, as well as the {\ttfamily aero\+\_\+diam} Net\+C\+DF dimension specified in the \mbox{\hyperlink{output_format_diam_bin_grid}{Output File Format\+: Diameter Bin Grid Data}} section.

The aerosol binned sectional state Net\+C\+DF variables are\+:
\begin{DoxyItemize}
\item {\bfseries aero\+\_\+number\+\_\+concentration} (unit 1/m$^\wedge$3, dim {\ttfamily aero\+\_\+diam})\+: the number size distribution for the aerosol population, $ dN(r)/d\ln r $, per bin
\item {\bfseries aero\+\_\+mass\+\_\+concentration} (unit kg/m$^\wedge$3, dim {\ttfamily dimid\+\_\+aero\+\_\+diam x dimid\+\_\+aero\+\_\+species})\+: the mass size distribution for the aerosol population, $ dM(r,s)/d\ln r $, per bin and per species 
\end{DoxyItemize}