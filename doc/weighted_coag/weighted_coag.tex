\documentclass{article}
\usepackage{amsmath}

\begin{document}

\section{Setup}

\begin{align}
  f &= \frac{W_1 W_2}{W_{\rm min}} \\
  W_i &= \frac{w_i}{V_i} \\
  &= \frac{1}{V_i} \left(\frac{v_i}{v_0}\right)^{\alpha_i} \\
  f &= \frac{\frac{1}{V_1} \left(\frac{v_1}{v_0}\right)^{\alpha_1}
    \frac{1}{V_2} \left(\frac{v_2}{v_0}\right)^{\alpha_2}}{
    \min\left\{ \frac{1}{V_1} \left(\frac{v_1}{v_0}\right)^{\alpha_1},
        \frac{1}{V_2} \left(\frac{v_2}{v_0}\right)^{\alpha_2},
        \frac{1}{V_{\rm c}} \left(\frac{v_1 + v_2}{v_0}\right)^{\alpha_{\rm c}} \right\}}
\end{align}

\section{All weightings the same}

$V = V_1 = V_2 = V_{\rm c}$ and $\alpha = \alpha_1 = \alpha_2 = \alpha_{\rm c}$:
\begin{align}
  f &= \frac{\frac{1}{V} \left(\frac{v_1}{v_0}\right)^{\alpha}
    \frac{1}{V} \left(\frac{v_2}{v_0}\right)^{\alpha}}{
    \min\left\{ \frac{1}{V} \left(\frac{v_1}{v_0}\right)^{\alpha},
        \frac{1}{V} \left(\frac{v_2}{v_0}\right)^{\alpha},
        \frac{1}{V} \left(\frac{v_1 + v_2}{v_0}\right)^{\alpha} \right\}}
\end{align}
$\alpha < 0$ implies minimum at $v_1 + v_2$:
\begin{align}
  f &= \frac{\frac{1}{V} \left(\frac{v_1}{v_0}\right)^{\alpha}
    \frac{1}{V} \left(\frac{v_2}{v_0}\right)^{\alpha}}{
    \frac{1}{V} \left(\frac{v_1 + v_2}{v_0}\right)^{\alpha}} \\
  &= \frac{1}{V v_0^\alpha} \left(\frac{v_1}{v_1 + v_2}\right)^\alpha v_2^\alpha \\
  &= \frac{1}{V v_0^\alpha} v_1^\alpha \left(\frac{v_2}{v_1 + v_2}\right)^\alpha
\end{align}
$f$ is decreasing in both $v_1$ and $v_2$

$\alpha > 0$ implies minimum at $v_1$ or $v_2$. Assume $v_1$ WLOG.
\begin{align}
  f &= \frac{\frac{1}{V} \left(\frac{v_1}{v_0}\right)^{\alpha}
    \frac{1}{V} \left(\frac{v_2}{v_0}\right)^{\alpha}}{
    \frac{1}{V} \left(\frac{v_1}{v_0}\right)^{\alpha}} \\
  &= \frac{1}{V} \left(\frac{v_2}{v_0}\right)^{\alpha}
\end{align}
so $f$ is increasing in both $v_1$ and $v_2$.

\section{Junk}

$V_{\rm c} = V_1$ and $\alpha_{\rm c} = \alpha_1$:
\begin{align}
  f &= \frac{\frac{1}{V_1} \left(\frac{v_1}{v_0}\right)^{\alpha_1}
    \frac{1}{V_2} \left(\frac{v_2}{v_0}\right)^{\alpha_2}}{
    \min\left\{ \frac{1}{V_1} \left(\frac{v_1}{v_0}\right)^{\alpha_1},
        \frac{1}{V_2} \left(\frac{v_2}{v_0}\right)^{\alpha_2},
        \frac{1}{V_1} \left(\frac{v_1 + v_2}{v_0}\right)^{\alpha_1} \right\}}
\end{align}
Minimum at $v_1$ implies $\alpha_1 > 0$ and
\begin{align}
  f &= \frac{1}{V_2} \left(\frac{v_2}{v_0}\right)^{\alpha_2}
\end{align}
so if $\alpha_2 > 0$ then $f$ is increasing in $v_2$, otherwise
decreasing.

Minimum at $v_2$ implies that if $\alpha_1 > 0$, $f$ is increasing in
$v_1$, otherwise it is decreasing.

Minimum at $v_1 + v_2$ implies $\alpha_1 < 0$ and
\begin{align}
  f &= \frac{\frac{1}{V_1} \left(\frac{v_1}{v_0}\right)^{\alpha_1}
    \frac{1}{V_2} \left(\frac{v_2}{v_0}\right)^{\alpha_2}}{
    \frac{1}{V_1} \left(\frac{v_1 + v_2}{v_0}\right)^{\alpha_1}} \\
  &= \left(\frac{v_1}{v_1 + v_2}\right)^{\alpha_1}
  \frac{1}{V_2} \left(\frac{v_2}{v_0}\right)^{\alpha_2}
\end{align}
so $f$ is increasing in $v_1$. Also,
\begin{align}
  f &= \frac{1}{V_2} \frac{v_1^{\alpha_1}}{v_0^{\alpha_2}}
  \frac{v_2^{\alpha_2}}{(v_1 + v_2)^{\alpha_1}} \\
  \frac{\partial f}{\partial v_2} &= \frac{1}{V_2}
  \frac{v_1^{\alpha_1}}{v_0^{\alpha_2}} \frac{\alpha_2
    v_2^{(\alpha_2-1)} (v_1+v_2)^{\alpha_1} - v_2^{\alpha_2} \alpha_1
    (v_1+v_2)^{(\alpha_1-1)}}{
    (v_1 + v_2)^{2\alpha_1}} \\
  &= f \left(\frac{\alpha_2}{v_2} - \frac{\alpha_1}{v_1+v_2}\right) \\
  &= f \left(\frac{\alpha_2 v_1 + (\alpha_2 - \alpha_1) v_2}{v_2(v_1 +
      v_2)} \right)
\end{align}
so if $\alpha_2 \ge \alpha_1$ then $f$ is increasing in $v_2$.

If $\alpha_2 < \alpha_1$ then a maximum value is achieved for
\begin{align}
  \alpha_2 v_1 + (\alpha_2 - \alpha_1) v_2 &= 0 \\
  v_2^* &= \frac{\alpha_2}{\alpha_1 - \alpha_2} v_1
\end{align}
If $v_2 \in [v_2^-,v_2^+]$ then if $v_2^* < v_2^-$, the maximum is at
$v_2^-$, if $v_2^* > v_2+$, then maximum is at $v_2^+$, otherwise the
maximum is at $v_2^*$.
\begin{align}
  v_2^{\rm max} &= \min(v_2^+, \max(v_2^-, v_2^*))
\end{align}

\section{All weightings different}

In general,
\begin{align}
  f &= \max\left\{
    \frac{1}{V_2} \left(\frac{v_2}{v_0}\right)^{\alpha_2},
    \frac{1}{V_1} \left(\frac{v_1}{v_0}\right)^{\alpha_1},
    \frac{\frac{1}{V_1} \left(\frac{v_1}{v_0}\right)^{\alpha_1}
      \frac{1}{V_2} \left(\frac{v_2}{v_0}\right)^{\alpha_2}}{
      \frac{1}{V_{\rm c}} \left(\frac{v_1 + v_2}{v_0}\right)^{\alpha_{\rm c}}}
  \right\}
\end{align}
The first two terms are easy, they are just monotonically increasing
or decreasing in $v_1$ and $v_2$, depending on the signs of $\alpha_1$
and $\alpha_2$. Let the third term be
\begin{align}
  T_3 &= \frac{\frac{1}{V_1} \left(\frac{v_1}{v_0}\right)^{\alpha_1}
    \frac{1}{V_2} \left(\frac{v_2}{v_0}\right)^{\alpha_2}}{
    \frac{1}{V_{\rm c}} \left(\frac{v_1 + v_2}{v_0}\right)^{\alpha_{\rm c}}}
\end{align}
Then
\begin{align}
  \frac{\partial T_3}{\partial v_1} &=
  T_3 \left(\frac{\alpha_1}{v_1} - \frac{\alpha_{\rm c}}{v_1+v_2}\right) \\
  &= T_3 \left(\frac{(\alpha_1 - \alpha_{\rm c}) v_1 + \alpha_1 v_2}{
      v_1 (v_1 + v_2)}\right)
\end{align}
so $T_3$ is monotonically increasing in $v_1$ if $\alpha_1 \ge
\alpha_{\rm c}$, otherwise a maximum occurs at
\begin{align}
  (\alpha_1 - \alpha_{\rm c}) v_1^* + \alpha_1 v_2 &= 0 \\
  v_1^* &= \frac{\alpha_1}{\alpha_{\rm c} - \alpha_1} v_2
\end{align}
If $v_1 \in [v_1^-,v_1^+]$ then if $v_1^* < v_1^-$, the maximum is at
$v_1^-$, if $v_1^* > v_1+$, then maximum is at $v_1^+$, otherwise the
maximum is at $v_1^*$.
\begin{align}
  v_1^{\rm max} &= \min(v_1^+, \max(v_1^-, v_1^*))
\end{align}
Thus,
\begin{align}
  v_1^* &= \frac{\alpha_1}{\alpha_{\rm c} - \alpha_1} v_2 \\
  v_1^{\rm max} &= \begin{cases}
    v_1^+ & \text{if } \alpha_1 \ge \alpha_{\rm c} \\
    \min(v_1^+, \max(v_1^-, v_1^*)) & \text{if } \alpha_1 < \alpha_{\rm c}
    \end{cases}
\end{align}
Similarly,
\begin{align}
  v_2^* &= \frac{\alpha_2}{\alpha_{\rm c} - \alpha_2} v_1 \\
  v_2^{\rm max} &= \begin{cases}
    v_2^+ & \text{if } \alpha_2 \ge \alpha_{\rm c} \\
    \min(v_2^+, \max(v_2^-, v_2^*)) & \text{if } \alpha_2 < \alpha_{\rm c}
    \end{cases}
\end{align}
In summary:
\begin{align}
  \alpha_1 \ge \alpha_c \text{ and } \alpha_2 \ge \alpha_c
  &\Longrightarrow v_1 = v_1^+ \text{ and } v_2 = v_2^+ \\
  \alpha_1 \ge \alpha_c \text{ and } \alpha_2 < \alpha_c
  &\Longrightarrow v_1 = v_1^+ \text{ and } v_2 = v_2^{\rm max}(v_1) \\
  \alpha_1 < \alpha_c \text{ and } \alpha_2 \ge \alpha_c
  &\Longrightarrow v_2 = v_2^+ \text{ and } v_1 = v_1^{\rm max}(v_2)  \\
  \alpha_1 < \alpha_c \text{ and } \alpha_2 < \alpha_c
  &\Longrightarrow \text{complicated}
\end{align}
We assume that $\min(\alpha_1,\alpha_2) \le \alpha_c \le
\max(\alpha_1,\alpha_2)$, so the last case above cannot occur.

\section{Minimum weight destination bin}

Assume $c = 1$ or $c = 2$ so that $W_c$ is minimized. Then
\begin{align}
  f &= \max\left\{
    \frac{1}{V_2} \left(\frac{v_2}{v_0}\right)^{\alpha_2},
    \frac{1}{V_1} \left(\frac{v_1}{v_0}\right)^{\alpha_1},
    \frac{\frac{1}{V_1} \left(\frac{v_1}{v_0}\right)^{\alpha_1}
      \frac{1}{V_2} \left(\frac{v_2}{v_0}\right)^{\alpha_2}}{
      \frac{1}{V_1} \left(\frac{v_1 + v_2}{v_0}\right)^{\alpha_1}},
    \frac{\frac{1}{V_1} \left(\frac{v_1}{v_0}\right)^{\alpha_1}
      \frac{1}{V_2} \left(\frac{v_2}{v_0}\right)^{\alpha_2}}{
      \frac{1}{V_2} \left(\frac{v_1 + v_2}{v_0}\right)^{\alpha_2}}
  \right\}
\end{align}
Let $f_i$ for $i = 1,\ldots,4$ be the four terms above. Handling $f_1$
and $f_2$ is easy, as they are obviously increasing or decreasing in
$v_1$ and $v_2$.

For $f_3$ we have
\begin{align}
  f_3 &= \frac{\frac{1}{V_1} \left(\frac{v_1}{v_0}\right)^{\alpha_1}
      \frac{1}{V_2} \left(\frac{v_2}{v_0}\right)^{\alpha_2}}{
      \frac{1}{V_1} \left(\frac{v_1 + v_2}{v_0}\right)^{\alpha_1}} \\
    &= \frac{1}{V_2} \left(\frac{v_2}{v_0}\right)^{\alpha_2}
    \left(\frac{v_1}{v_1 + v_2}\right)^{\alpha_1}
\end{align}
This is increasing in $v_1$ (if $\alpha_1 > 0$) or decreasing in $v_1$
(if $\alpha_1 < 0$). To see the behavior in $v_2$ we compute
\begin{align}
  \frac{\partial f_3}{\partial v_2}
  &= f_3 \left(\frac{\alpha_2 v_1 + (\alpha_2 - \alpha_1) v_2}{
      v_2 (v_1 + v_2)}\right)
\end{align}
If $\alpha_2 \ge 0$ then $f_3$ is strictly increasing in $v_2$. If
$\alpha_2 < \alpha_1$ then we define
\begin{align}
  v_2^* &= \frac{\alpha_2}{\alpha_1 - \alpha_2} v_1 \\
  v_2^{\rm max} &= \min(v_2^+, \max(v_2^-, v_2^*))
\end{align}
Note that if $\alpha_1 \le 0$ then $f_3 \ge f_1$ (and the reverse if
$\alpha_1 \ge 0$), so we could just compute $f_3$ and $f_4$ if we know
$\alpha_1,\alpha_2 \le 0$.

\section{Accelerated coagulation}

Event probabilities are
\begin{align}
  p_{\text{remove},1} &= \frac{W_1}{W_{\rm min}} \\
  p_{\text{remove},2} &= \frac{W_2}{W_{\rm min}} \\
  p_{\text{create}} &= \frac{W_{\rm c}}{W_{\rm min}}
\end{align}
The target particle should be the one with the lowest
$p_{\text{remove}}$, and the other one should be the sampled
partner. Assume WLOG that the target particle is $1$ and the sampled
partner is $2$.

Assume that $p_{\text{create}} = 1$.

\end{document}
